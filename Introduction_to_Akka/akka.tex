\documentclass{beamer}

\usetheme{Singapore}

\usepackage{listings}

\title{Introduction to Akka}
\author{Lenko Donchev}

\begin{document}

\frame{\titlepage}

\section[Outline]{}
\frame{\tableofcontents}

\section{Introduction}
\subsection{Akka overview}

\frame
    {
      \frametitle{What is Akka?}
      
      \begin{itemize}
      \item Event-driven middleware framework for building high
        performance and reliable distributed applications in Scala and Java.
      \item  Open Source and available under the Apache 2 License.
      \end{itemize}
    }

\frame
    {
      \frametitle{Why Akka?}
      
      \begin{itemize}
      \item Scalability
      \item Concurrency
      \item Fault-Tolerance
      \end{itemize}
    }

\frame
    {
      \frametitle{Simple Concurrency Distribution}

      \begin{itemize}
        \item Asynchronous and Distributed by design.
        \item High-level abstractions like Actors, Futures and STM
      \end{itemize}
    }



\frame
    {
      \frametitle{Resilient by Design}

      \begin{itemize}
        \item Write systems that self-heal. 
        \item Remote and/or local supervisor hierarchies.
      \end{itemize}
    }


\frame
    {
      \frametitle{High Performance}

      \begin{itemize}
        \item 50 million msg/sec on a single machine. 
        \item Small memory footprint; ~2.5 million actors per GB of heap.. 
      \end{itemize}
    }


\frame
    {
      \frametitle{Elastic Decentralized}

      \begin{itemize}
        \item  Adaptive load balancing, routing, partitioning and configuration-driven remoting.
      \end{itemize}
    }


\frame
    {
      \frametitle{Extensible}

      \begin{itemize}
        \item Use Akka Extensions to adapt Akka to fit your needs.
      \end{itemize}
    }




    \frame
        {
          \frametitle{STM - Software Transactional Memory}
          \begin{itemize}
          \item Turns the JVM heap into a transactional data set
          \item Provides begin/commit/rollback semantics
          \item Implements the first three letters of ACID
            atomicity - all or none
            consistency - data is left in a consistent state
            isolation - only participants see changes during transaction
          \item Not Durable because STM is in-memory
          \item Modeled After Clojure's STM
          \item Transactions are automatically retried on collisions
          \end{itemize}
        }



\frame
    {
      \frametitle{Concurrency with Actors}
      \begin{itemize}
      \item Provides a high-level abstraction for concurrent and distributed
        system development
      \item  Asynchronous message processing using event-driven
        receive loop
      \item Removes the burden of explicit thread and lock management to
        make concurrent programming easier
      \item Pattern matching
        against messages is a convenient way to express an actor's
        behavior.
      \item Very lightweight
      \item Helps you to focus on the message workflow instead of
        low level primitives like threads, locks and socket IO
      \end{itemize}
    }



\frame
    {
      \frametitle{The origin of Actors}
      \begin{itemize}
      \item Defined in a 1973 paper by Carl Hewitt

      \item Popularized by Erlang

      \item originally developed at Ericsson
      \item  designed for distributed, fault-tolerant, non-stop systems
      \item 9-nine's reliability or down-time of 31 ms/year
      \item direct support for actor concurrency model in the language
      \item supports hot-swapping of code
      \end{itemize}
    }




\frame
{
  \frametitle{An Actor...}

  \begin{itemize}
  \item Encapsulates state and behavior into a lightweight "process"
  \item Supports Hot Swapping of the Actor’s message loop (e.g. its implementation) at runtime. 
  \item Shares nothing with other actors
  \item Communicates with other actors through messages 
  \item Communicates asynchronously
  \item Has a message queue or "mailbox"
  \item Has support for durable mailboxes
  \item Non-blocking
  \end{itemize}
}

\frame
    {
      \frametitle{Advantages of the Actor Model}
      \begin{itemize}
      \item Is easier to reason about
      \item Raises the level of abstraction
      \item Makes it easier to avoid:
        \begin{itemize}
        \item race conditions
        \item deadlocks
        \item starvation
        \item live locks
        \end{itemize}
      \end{itemize}
    }



\frame
{
  \frametitle{Remote Actors}
  \begin{itemize}
  \item Remote Actors provide a way to scale "out"
  \item Actors are excellent for distributed computing
  \item Remote Actors are implemented with NIO on top of
    \begin{itemize}
    \item JBoss Netty; an NIO client server framework
    \item Google Protcol Buffers; structured data encoding format
    \end{itemize}
  \end{itemize}
}


\frame
    {
      \frametitle{Fault Tolerance}
      \begin{itemize}
      \item The "let it crash" approach
      \item Designed for concurrent and distributed systems
      \item Notification of failures
      \item Supervision and repair of failed nodes
      \end{itemize}
    }


\frame
{
  \frametitle{Accept Failure as a fact of life and manage it}
  Built in Fault-tolerant design from the ground up
  \begin{itemize}
    \item Supervisor Hierarchies have a different view of failure
    \item Components are monitored by a "linked" supervisor
        When the supervisor detects failure, nodes are:
         \begin{itemize}
           \item reset to a stable state
           \item restarted
         \end{itemize}
  \end{itemize}
}


\frame
{
  \frametitle{Restart Strategies}
  Akka supports two restart strategies
  \begin{itemize}
  \item OneForOne - restarts the component that crashed
  \item AllForOne - restarts all managed components if one crashed
  \end{itemize}
}

\frame
{
  \frametitle{OneForOne Strategy}
  \begin{itemize}
  \item Failure of one actor forces restart of only that actor
    \begin{itemize}
       \item Actor Fails, throwing exception
       \item Exception thrown by actor is propagated to Supervisor
       \item Supervising Actor restarts failed actor
    \end{itemize}
  \item Restart initializes actor to a well-known state
  \end{itemize}
}

\frame
{
  \frametitle{AllForOne Strategy}
  \begin{itemize}
  \item Failure of one actor forces restart of all supervised actors
    \begin{itemize}
       \item Actor Fails, throwing exception
       \item Exception thrown by actor is propagated to Supervisor
       \item Supervising Actor restarts all supervised actors
    \end{itemize}
  \item Restart initializes actors to a well-known state
  \end{itemize}
}

\frame
{
  \frametitle{Restart Callbacks}
  \begin{itemize}
  \item Actors have 2 different restart callbacks
    \begin{itemize}
       \item pre restart
       \item post restart
    \end{itemize}
  \item Used to clean up and reinitialize state upon restart
  \end{itemize}
}


\frame
{
  \frametitle{Add-On Modules}
  \begin{itemize}
  \item Persistence
  \item STM integration with NoSQL DBs for the 'D' in ACID
    Cassandra, MongoDB, Redis, CouchDB, Amazon SimpleDB,
    and others
  \item AMQP - based on the RabbitMQ client
  \item JTA - allows STM to participate in a JTA transaction
  \item Spring Integration
  \item Guice Integration
  \end{itemize}
}


\section{Examples}
\subsection{Example Akka actors}

\begin{frame}[fragile]
  \frametitle{Example Actor in Scala}

  \begin{lstlisting}
    class HelloWorldActor extends Actor {
      def receive = {
        case "hello" => println("Hello World!")
        case _ => println("hi")
      }
    }
  \end{lstlisting}
\end{frame}


\begin{frame}[fragile]
  \frametitle{Example Actor in Java}

  \begin{lstlisting}
    public class HelloWorldActor extends UntypedActor {
      public void onReceive(Object message) throws Exception {
        if (message instanceof String)
         System.out.println("Hello world...");
        else
         System.out.println("hi");
      }
    }
  \end{lstlisting}
\end{frame}



\begin{frame}[fragile]
  \frametitle{Asynchronous Messaging in Java}

  \begin{lstlisting}
    helloWorldActor.sendOneWay("Hello");
  \end{lstlisting}
\end{frame}



\begin{frame}[fragile]
  \frametitle{Synchronous Messaging in Java}

  \begin{lstlisting}
    // send and receive
    try {
      Object result =
      helloWorldActor.sendRequestReply("Hello", getContext(), 1000);
    } catch(ActorTimeoutException ate) {
      // handle timeout
    }
  \end{lstlisting}
\end{frame}


\begin{frame}[fragile]
  \frametitle{Replying to Messages}

  \begin{lstlisting}
    self.reply("reply")

    // Scala
    self.sender.get ! "reply"

  \end{lstlisting}
\end{frame}



\begin{frame}[fragile]
  \frametitle{Asynchronous Messaging in Scala}

  \begin{lstlisting}
    // fire-and-forget syntax (asynchronous)
    helloWorldActor ! "Hello"
    // returns immediately

  \end{lstlisting}
\end{frame}


\begin{frame}[fragile]
  \frametitle{Synchronous Messaging in Scala}

  \begin{lstlisting}
    // Send-and-Receive syntax (synchronous)
    helloWorldActor !! "Hello"
  \end{lstlisting}
\end{frame}


\frame
{
  \frametitle{Example: Building a  simple Distributed Search Engine}
  \href{https://github.com/lenko-d/distributed\_search\_engine\_in\_akka}{https://github.com/lenko-d/distributed\_search\_engine}
}



\section{About}

\frame
{
	\frametitle{About}
        \begin{center}
        Lenko Donchev 
        \end{center}
        \begin{center}
        \href{http://twitter.com/Lenko\_HD}{ @Twitter: Lenko\_HD}
        \end{center}
}

\end{document}
